\documentclass[11pt]{article}

    \usepackage[breakable]{tcolorbox}
    \usepackage{parskip} % Stop auto-indenting (to mimic markdown behaviour)
    
    \usepackage{iftex}
    \ifPDFTeX
    	\usepackage[T1]{fontenc}
    	\usepackage{mathpazo}
    \else
    	\usepackage{fontspec}
    \fi

    % Basic figure setup, for now with no caption control since it's done
    % automatically by Pandoc (which extracts ![](path) syntax from Markdown).
    \usepackage{graphicx}
    % Maintain compatibility with old templates. Remove in nbconvert 6.0
    \let\Oldincludegraphics\includegraphics
    % Ensure that by default, figures have no caption (until we provide a
    % proper Figure object with a Caption API and a way to capture that
    % in the conversion process - todo).
    \usepackage{caption}
    \DeclareCaptionFormat{nocaption}{}
    \captionsetup{format=nocaption,aboveskip=0pt,belowskip=0pt}

    \usepackage{float}
    \floatplacement{figure}{H} % forces figures to be placed at the correct location
    \usepackage{xcolor} % Allow colors to be defined
    \usepackage{enumerate} % Needed for markdown enumerations to work
    \usepackage{geometry} % Used to adjust the document margins
    \usepackage{amsmath} % Equations
    \usepackage{amssymb} % Equations
    \usepackage{textcomp} % defines textquotesingle
    % Hack from http://tex.stackexchange.com/a/47451/13684:
    \AtBeginDocument{%
        \def\PYZsq{\textquotesingle}% Upright quotes in Pygmentized code
    }
    \usepackage{upquote} % Upright quotes for verbatim code
    \usepackage{eurosym} % defines \euro
    \usepackage[mathletters]{ucs} % Extended unicode (utf-8) support
    \usepackage{fancyvrb} % verbatim replacement that allows latex
    \usepackage{grffile} % extends the file name processing of package graphics 
                         % to support a larger range
    \makeatletter % fix for old versions of grffile with XeLaTeX
    \@ifpackagelater{grffile}{2019/11/01}
    {
      % Do nothing on new versions
    }
    {
      \def\Gread@@xetex#1{%
        \IfFileExists{"\Gin@base".bb}%
        {\Gread@eps{\Gin@base.bb}}%
        {\Gread@@xetex@aux#1}%
      }
    }
    \makeatother
    \usepackage[Export]{adjustbox} % Used to constrain images to a maximum size
    \adjustboxset{max size={0.9\linewidth}{0.9\paperheight}}

    % The hyperref package gives us a pdf with properly built
    % internal navigation ('pdf bookmarks' for the table of contents,
    % internal cross-reference links, web links for URLs, etc.)
    \usepackage{hyperref}
    % The default LaTeX title has an obnoxious amount of whitespace. By default,
    % titling removes some of it. It also provides customization options.
    \usepackage{titling}
    \usepackage{longtable} % longtable support required by pandoc >1.10
    \usepackage{booktabs}  % table support for pandoc > 1.12.2
    \usepackage[inline]{enumitem} % IRkernel/repr support (it uses the enumerate* environment)
    \usepackage[normalem]{ulem} % ulem is needed to support strikethroughs (\sout)
                                % normalem makes italics be italics, not underlines
    \usepackage{mathrsfs}
    

    
    % Colors for the hyperref package
    \definecolor{urlcolor}{rgb}{0,.145,.698}
    \definecolor{linkcolor}{rgb}{.71,0.21,0.01}
    \definecolor{citecolor}{rgb}{.12,.54,.11}

    % ANSI colors
    \definecolor{ansi-black}{HTML}{3E424D}
    \definecolor{ansi-black-intense}{HTML}{282C36}
    \definecolor{ansi-red}{HTML}{E75C58}
    \definecolor{ansi-red-intense}{HTML}{B22B31}
    \definecolor{ansi-green}{HTML}{00A250}
    \definecolor{ansi-green-intense}{HTML}{007427}
    \definecolor{ansi-yellow}{HTML}{DDB62B}
    \definecolor{ansi-yellow-intense}{HTML}{B27D12}
    \definecolor{ansi-blue}{HTML}{208FFB}
    \definecolor{ansi-blue-intense}{HTML}{0065CA}
    \definecolor{ansi-magenta}{HTML}{D160C4}
    \definecolor{ansi-magenta-intense}{HTML}{A03196}
    \definecolor{ansi-cyan}{HTML}{60C6C8}
    \definecolor{ansi-cyan-intense}{HTML}{258F8F}
    \definecolor{ansi-white}{HTML}{C5C1B4}
    \definecolor{ansi-white-intense}{HTML}{A1A6B2}
    \definecolor{ansi-default-inverse-fg}{HTML}{FFFFFF}
    \definecolor{ansi-default-inverse-bg}{HTML}{000000}

    % common color for the border for error outputs.
    \definecolor{outerrorbackground}{HTML}{FFDFDF}

    % commands and environments needed by pandoc snippets
    % extracted from the output of `pandoc -s`
    \providecommand{\tightlist}{%
      \setlength{\itemsep}{0pt}\setlength{\parskip}{0pt}}
    \DefineVerbatimEnvironment{Highlighting}{Verbatim}{commandchars=\\\{\}}
    % Add ',fontsize=\small' for more characters per line
    \newenvironment{Shaded}{}{}
    \newcommand{\KeywordTok}[1]{\textcolor[rgb]{0.00,0.44,0.13}{\textbf{{#1}}}}
    \newcommand{\DataTypeTok}[1]{\textcolor[rgb]{0.56,0.13,0.00}{{#1}}}
    \newcommand{\DecValTok}[1]{\textcolor[rgb]{0.25,0.63,0.44}{{#1}}}
    \newcommand{\BaseNTok}[1]{\textcolor[rgb]{0.25,0.63,0.44}{{#1}}}
    \newcommand{\FloatTok}[1]{\textcolor[rgb]{0.25,0.63,0.44}{{#1}}}
    \newcommand{\CharTok}[1]{\textcolor[rgb]{0.25,0.44,0.63}{{#1}}}
    \newcommand{\StringTok}[1]{\textcolor[rgb]{0.25,0.44,0.63}{{#1}}}
    \newcommand{\CommentTok}[1]{\textcolor[rgb]{0.38,0.63,0.69}{\textit{{#1}}}}
    \newcommand{\OtherTok}[1]{\textcolor[rgb]{0.00,0.44,0.13}{{#1}}}
    \newcommand{\AlertTok}[1]{\textcolor[rgb]{1.00,0.00,0.00}{\textbf{{#1}}}}
    \newcommand{\FunctionTok}[1]{\textcolor[rgb]{0.02,0.16,0.49}{{#1}}}
    \newcommand{\RegionMarkerTok}[1]{{#1}}
    \newcommand{\ErrorTok}[1]{\textcolor[rgb]{1.00,0.00,0.00}{\textbf{{#1}}}}
    \newcommand{\NormalTok}[1]{{#1}}
    
    % Additional commands for more recent versions of Pandoc
    \newcommand{\ConstantTok}[1]{\textcolor[rgb]{0.53,0.00,0.00}{{#1}}}
    \newcommand{\SpecialCharTok}[1]{\textcolor[rgb]{0.25,0.44,0.63}{{#1}}}
    \newcommand{\VerbatimStringTok}[1]{\textcolor[rgb]{0.25,0.44,0.63}{{#1}}}
    \newcommand{\SpecialStringTok}[1]{\textcolor[rgb]{0.73,0.40,0.53}{{#1}}}
    \newcommand{\ImportTok}[1]{{#1}}
    \newcommand{\DocumentationTok}[1]{\textcolor[rgb]{0.73,0.13,0.13}{\textit{{#1}}}}
    \newcommand{\AnnotationTok}[1]{\textcolor[rgb]{0.38,0.63,0.69}{\textbf{\textit{{#1}}}}}
    \newcommand{\CommentVarTok}[1]{\textcolor[rgb]{0.38,0.63,0.69}{\textbf{\textit{{#1}}}}}
    \newcommand{\VariableTok}[1]{\textcolor[rgb]{0.10,0.09,0.49}{{#1}}}
    \newcommand{\ControlFlowTok}[1]{\textcolor[rgb]{0.00,0.44,0.13}{\textbf{{#1}}}}
    \newcommand{\OperatorTok}[1]{\textcolor[rgb]{0.40,0.40,0.40}{{#1}}}
    \newcommand{\BuiltInTok}[1]{{#1}}
    \newcommand{\ExtensionTok}[1]{{#1}}
    \newcommand{\PreprocessorTok}[1]{\textcolor[rgb]{0.74,0.48,0.00}{{#1}}}
    \newcommand{\AttributeTok}[1]{\textcolor[rgb]{0.49,0.56,0.16}{{#1}}}
    \newcommand{\InformationTok}[1]{\textcolor[rgb]{0.38,0.63,0.69}{\textbf{\textit{{#1}}}}}
    \newcommand{\WarningTok}[1]{\textcolor[rgb]{0.38,0.63,0.69}{\textbf{\textit{{#1}}}}}
    
    
    % Define a nice break command that doesn't care if a line doesn't already
    % exist.
    \def\br{\hspace*{\fill} \\* }
    % Math Jax compatibility definitions
    \def\gt{>}
    \def\lt{<}
    \let\Oldtex\TeX
    \let\Oldlatex\LaTeX
    \renewcommand{\TeX}{\textrm{\Oldtex}}
    \renewcommand{\LaTeX}{\textrm{\Oldlatex}}
    % Document parameters
    % Document title
    \title{Homework2}
    
    
    
    
    
% Pygments definitions
\makeatletter
\def\PY@reset{\let\PY@it=\relax \let\PY@bf=\relax%
    \let\PY@ul=\relax \let\PY@tc=\relax%
    \let\PY@bc=\relax \let\PY@ff=\relax}
\def\PY@tok#1{\csname PY@tok@#1\endcsname}
\def\PY@toks#1+{\ifx\relax#1\empty\else%
    \PY@tok{#1}\expandafter\PY@toks\fi}
\def\PY@do#1{\PY@bc{\PY@tc{\PY@ul{%
    \PY@it{\PY@bf{\PY@ff{#1}}}}}}}
\def\PY#1#2{\PY@reset\PY@toks#1+\relax+\PY@do{#2}}

\@namedef{PY@tok@w}{\def\PY@tc##1{\textcolor[rgb]{0.73,0.73,0.73}{##1}}}
\@namedef{PY@tok@c}{\let\PY@it=\textit\def\PY@tc##1{\textcolor[rgb]{0.25,0.50,0.50}{##1}}}
\@namedef{PY@tok@cp}{\def\PY@tc##1{\textcolor[rgb]{0.74,0.48,0.00}{##1}}}
\@namedef{PY@tok@k}{\let\PY@bf=\textbf\def\PY@tc##1{\textcolor[rgb]{0.00,0.50,0.00}{##1}}}
\@namedef{PY@tok@kp}{\def\PY@tc##1{\textcolor[rgb]{0.00,0.50,0.00}{##1}}}
\@namedef{PY@tok@kt}{\def\PY@tc##1{\textcolor[rgb]{0.69,0.00,0.25}{##1}}}
\@namedef{PY@tok@o}{\def\PY@tc##1{\textcolor[rgb]{0.40,0.40,0.40}{##1}}}
\@namedef{PY@tok@ow}{\let\PY@bf=\textbf\def\PY@tc##1{\textcolor[rgb]{0.67,0.13,1.00}{##1}}}
\@namedef{PY@tok@nb}{\def\PY@tc##1{\textcolor[rgb]{0.00,0.50,0.00}{##1}}}
\@namedef{PY@tok@nf}{\def\PY@tc##1{\textcolor[rgb]{0.00,0.00,1.00}{##1}}}
\@namedef{PY@tok@nc}{\let\PY@bf=\textbf\def\PY@tc##1{\textcolor[rgb]{0.00,0.00,1.00}{##1}}}
\@namedef{PY@tok@nn}{\let\PY@bf=\textbf\def\PY@tc##1{\textcolor[rgb]{0.00,0.00,1.00}{##1}}}
\@namedef{PY@tok@ne}{\let\PY@bf=\textbf\def\PY@tc##1{\textcolor[rgb]{0.82,0.25,0.23}{##1}}}
\@namedef{PY@tok@nv}{\def\PY@tc##1{\textcolor[rgb]{0.10,0.09,0.49}{##1}}}
\@namedef{PY@tok@no}{\def\PY@tc##1{\textcolor[rgb]{0.53,0.00,0.00}{##1}}}
\@namedef{PY@tok@nl}{\def\PY@tc##1{\textcolor[rgb]{0.63,0.63,0.00}{##1}}}
\@namedef{PY@tok@ni}{\let\PY@bf=\textbf\def\PY@tc##1{\textcolor[rgb]{0.60,0.60,0.60}{##1}}}
\@namedef{PY@tok@na}{\def\PY@tc##1{\textcolor[rgb]{0.49,0.56,0.16}{##1}}}
\@namedef{PY@tok@nt}{\let\PY@bf=\textbf\def\PY@tc##1{\textcolor[rgb]{0.00,0.50,0.00}{##1}}}
\@namedef{PY@tok@nd}{\def\PY@tc##1{\textcolor[rgb]{0.67,0.13,1.00}{##1}}}
\@namedef{PY@tok@s}{\def\PY@tc##1{\textcolor[rgb]{0.73,0.13,0.13}{##1}}}
\@namedef{PY@tok@sd}{\let\PY@it=\textit\def\PY@tc##1{\textcolor[rgb]{0.73,0.13,0.13}{##1}}}
\@namedef{PY@tok@si}{\let\PY@bf=\textbf\def\PY@tc##1{\textcolor[rgb]{0.73,0.40,0.53}{##1}}}
\@namedef{PY@tok@se}{\let\PY@bf=\textbf\def\PY@tc##1{\textcolor[rgb]{0.73,0.40,0.13}{##1}}}
\@namedef{PY@tok@sr}{\def\PY@tc##1{\textcolor[rgb]{0.73,0.40,0.53}{##1}}}
\@namedef{PY@tok@ss}{\def\PY@tc##1{\textcolor[rgb]{0.10,0.09,0.49}{##1}}}
\@namedef{PY@tok@sx}{\def\PY@tc##1{\textcolor[rgb]{0.00,0.50,0.00}{##1}}}
\@namedef{PY@tok@m}{\def\PY@tc##1{\textcolor[rgb]{0.40,0.40,0.40}{##1}}}
\@namedef{PY@tok@gh}{\let\PY@bf=\textbf\def\PY@tc##1{\textcolor[rgb]{0.00,0.00,0.50}{##1}}}
\@namedef{PY@tok@gu}{\let\PY@bf=\textbf\def\PY@tc##1{\textcolor[rgb]{0.50,0.00,0.50}{##1}}}
\@namedef{PY@tok@gd}{\def\PY@tc##1{\textcolor[rgb]{0.63,0.00,0.00}{##1}}}
\@namedef{PY@tok@gi}{\def\PY@tc##1{\textcolor[rgb]{0.00,0.63,0.00}{##1}}}
\@namedef{PY@tok@gr}{\def\PY@tc##1{\textcolor[rgb]{1.00,0.00,0.00}{##1}}}
\@namedef{PY@tok@ge}{\let\PY@it=\textit}
\@namedef{PY@tok@gs}{\let\PY@bf=\textbf}
\@namedef{PY@tok@gp}{\let\PY@bf=\textbf\def\PY@tc##1{\textcolor[rgb]{0.00,0.00,0.50}{##1}}}
\@namedef{PY@tok@go}{\def\PY@tc##1{\textcolor[rgb]{0.53,0.53,0.53}{##1}}}
\@namedef{PY@tok@gt}{\def\PY@tc##1{\textcolor[rgb]{0.00,0.27,0.87}{##1}}}
\@namedef{PY@tok@err}{\def\PY@bc##1{{\setlength{\fboxsep}{\string -\fboxrule}\fcolorbox[rgb]{1.00,0.00,0.00}{1,1,1}{\strut ##1}}}}
\@namedef{PY@tok@kc}{\let\PY@bf=\textbf\def\PY@tc##1{\textcolor[rgb]{0.00,0.50,0.00}{##1}}}
\@namedef{PY@tok@kd}{\let\PY@bf=\textbf\def\PY@tc##1{\textcolor[rgb]{0.00,0.50,0.00}{##1}}}
\@namedef{PY@tok@kn}{\let\PY@bf=\textbf\def\PY@tc##1{\textcolor[rgb]{0.00,0.50,0.00}{##1}}}
\@namedef{PY@tok@kr}{\let\PY@bf=\textbf\def\PY@tc##1{\textcolor[rgb]{0.00,0.50,0.00}{##1}}}
\@namedef{PY@tok@bp}{\def\PY@tc##1{\textcolor[rgb]{0.00,0.50,0.00}{##1}}}
\@namedef{PY@tok@fm}{\def\PY@tc##1{\textcolor[rgb]{0.00,0.00,1.00}{##1}}}
\@namedef{PY@tok@vc}{\def\PY@tc##1{\textcolor[rgb]{0.10,0.09,0.49}{##1}}}
\@namedef{PY@tok@vg}{\def\PY@tc##1{\textcolor[rgb]{0.10,0.09,0.49}{##1}}}
\@namedef{PY@tok@vi}{\def\PY@tc##1{\textcolor[rgb]{0.10,0.09,0.49}{##1}}}
\@namedef{PY@tok@vm}{\def\PY@tc##1{\textcolor[rgb]{0.10,0.09,0.49}{##1}}}
\@namedef{PY@tok@sa}{\def\PY@tc##1{\textcolor[rgb]{0.73,0.13,0.13}{##1}}}
\@namedef{PY@tok@sb}{\def\PY@tc##1{\textcolor[rgb]{0.73,0.13,0.13}{##1}}}
\@namedef{PY@tok@sc}{\def\PY@tc##1{\textcolor[rgb]{0.73,0.13,0.13}{##1}}}
\@namedef{PY@tok@dl}{\def\PY@tc##1{\textcolor[rgb]{0.73,0.13,0.13}{##1}}}
\@namedef{PY@tok@s2}{\def\PY@tc##1{\textcolor[rgb]{0.73,0.13,0.13}{##1}}}
\@namedef{PY@tok@sh}{\def\PY@tc##1{\textcolor[rgb]{0.73,0.13,0.13}{##1}}}
\@namedef{PY@tok@s1}{\def\PY@tc##1{\textcolor[rgb]{0.73,0.13,0.13}{##1}}}
\@namedef{PY@tok@mb}{\def\PY@tc##1{\textcolor[rgb]{0.40,0.40,0.40}{##1}}}
\@namedef{PY@tok@mf}{\def\PY@tc##1{\textcolor[rgb]{0.40,0.40,0.40}{##1}}}
\@namedef{PY@tok@mh}{\def\PY@tc##1{\textcolor[rgb]{0.40,0.40,0.40}{##1}}}
\@namedef{PY@tok@mi}{\def\PY@tc##1{\textcolor[rgb]{0.40,0.40,0.40}{##1}}}
\@namedef{PY@tok@il}{\def\PY@tc##1{\textcolor[rgb]{0.40,0.40,0.40}{##1}}}
\@namedef{PY@tok@mo}{\def\PY@tc##1{\textcolor[rgb]{0.40,0.40,0.40}{##1}}}
\@namedef{PY@tok@ch}{\let\PY@it=\textit\def\PY@tc##1{\textcolor[rgb]{0.25,0.50,0.50}{##1}}}
\@namedef{PY@tok@cm}{\let\PY@it=\textit\def\PY@tc##1{\textcolor[rgb]{0.25,0.50,0.50}{##1}}}
\@namedef{PY@tok@cpf}{\let\PY@it=\textit\def\PY@tc##1{\textcolor[rgb]{0.25,0.50,0.50}{##1}}}
\@namedef{PY@tok@c1}{\let\PY@it=\textit\def\PY@tc##1{\textcolor[rgb]{0.25,0.50,0.50}{##1}}}
\@namedef{PY@tok@cs}{\let\PY@it=\textit\def\PY@tc##1{\textcolor[rgb]{0.25,0.50,0.50}{##1}}}

\def\PYZbs{\char`\\}
\def\PYZus{\char`\_}
\def\PYZob{\char`\{}
\def\PYZcb{\char`\}}
\def\PYZca{\char`\^}
\def\PYZam{\char`\&}
\def\PYZlt{\char`\<}
\def\PYZgt{\char`\>}
\def\PYZsh{\char`\#}
\def\PYZpc{\char`\%}
\def\PYZdl{\char`\$}
\def\PYZhy{\char`\-}
\def\PYZsq{\char`\'}
\def\PYZdq{\char`\"}
\def\PYZti{\char`\~}
% for compatibility with earlier versions
\def\PYZat{@}
\def\PYZlb{[}
\def\PYZrb{]}
\makeatother


    % For linebreaks inside Verbatim environment from package fancyvrb. 
    \makeatletter
        \newbox\Wrappedcontinuationbox 
        \newbox\Wrappedvisiblespacebox 
        \newcommand*\Wrappedvisiblespace {\textcolor{red}{\textvisiblespace}} 
        \newcommand*\Wrappedcontinuationsymbol {\textcolor{red}{\llap{\tiny$\m@th\hookrightarrow$}}} 
        \newcommand*\Wrappedcontinuationindent {3ex } 
        \newcommand*\Wrappedafterbreak {\kern\Wrappedcontinuationindent\copy\Wrappedcontinuationbox} 
        % Take advantage of the already applied Pygments mark-up to insert 
        % potential linebreaks for TeX processing. 
        %        {, <, #, %, $, ' and ": go to next line. 
        %        _, }, ^, &, >, - and ~: stay at end of broken line. 
        % Use of \textquotesingle for straight quote. 
        \newcommand*\Wrappedbreaksatspecials {% 
            \def\PYGZus{\discretionary{\char`\_}{\Wrappedafterbreak}{\char`\_}}% 
            \def\PYGZob{\discretionary{}{\Wrappedafterbreak\char`\{}{\char`\{}}% 
            \def\PYGZcb{\discretionary{\char`\}}{\Wrappedafterbreak}{\char`\}}}% 
            \def\PYGZca{\discretionary{\char`\^}{\Wrappedafterbreak}{\char`\^}}% 
            \def\PYGZam{\discretionary{\char`\&}{\Wrappedafterbreak}{\char`\&}}% 
            \def\PYGZlt{\discretionary{}{\Wrappedafterbreak\char`\<}{\char`\<}}% 
            \def\PYGZgt{\discretionary{\char`\>}{\Wrappedafterbreak}{\char`\>}}% 
            \def\PYGZsh{\discretionary{}{\Wrappedafterbreak\char`\#}{\char`\#}}% 
            \def\PYGZpc{\discretionary{}{\Wrappedafterbreak\char`\%}{\char`\%}}% 
            \def\PYGZdl{\discretionary{}{\Wrappedafterbreak\char`\$}{\char`\$}}% 
            \def\PYGZhy{\discretionary{\char`\-}{\Wrappedafterbreak}{\char`\-}}% 
            \def\PYGZsq{\discretionary{}{\Wrappedafterbreak\textquotesingle}{\textquotesingle}}% 
            \def\PYGZdq{\discretionary{}{\Wrappedafterbreak\char`\"}{\char`\"}}% 
            \def\PYGZti{\discretionary{\char`\~}{\Wrappedafterbreak}{\char`\~}}% 
        } 
        % Some characters . , ; ? ! / are not pygmentized. 
        % This macro makes them "active" and they will insert potential linebreaks 
        \newcommand*\Wrappedbreaksatpunct {% 
            \lccode`\~`\.\lowercase{\def~}{\discretionary{\hbox{\char`\.}}{\Wrappedafterbreak}{\hbox{\char`\.}}}% 
            \lccode`\~`\,\lowercase{\def~}{\discretionary{\hbox{\char`\,}}{\Wrappedafterbreak}{\hbox{\char`\,}}}% 
            \lccode`\~`\;\lowercase{\def~}{\discretionary{\hbox{\char`\;}}{\Wrappedafterbreak}{\hbox{\char`\;}}}% 
            \lccode`\~`\:\lowercase{\def~}{\discretionary{\hbox{\char`\:}}{\Wrappedafterbreak}{\hbox{\char`\:}}}% 
            \lccode`\~`\?\lowercase{\def~}{\discretionary{\hbox{\char`\?}}{\Wrappedafterbreak}{\hbox{\char`\?}}}% 
            \lccode`\~`\!\lowercase{\def~}{\discretionary{\hbox{\char`\!}}{\Wrappedafterbreak}{\hbox{\char`\!}}}% 
            \lccode`\~`\/\lowercase{\def~}{\discretionary{\hbox{\char`\/}}{\Wrappedafterbreak}{\hbox{\char`\/}}}% 
            \catcode`\.\active
            \catcode`\,\active 
            \catcode`\;\active
            \catcode`\:\active
            \catcode`\?\active
            \catcode`\!\active
            \catcode`\/\active 
            \lccode`\~`\~ 	
        }
    \makeatother

    \let\OriginalVerbatim=\Verbatim
    \makeatletter
    \renewcommand{\Verbatim}[1][1]{%
        %\parskip\z@skip
        \sbox\Wrappedcontinuationbox {\Wrappedcontinuationsymbol}%
        \sbox\Wrappedvisiblespacebox {\FV@SetupFont\Wrappedvisiblespace}%
        \def\FancyVerbFormatLine ##1{\hsize\linewidth
            \vtop{\raggedright\hyphenpenalty\z@\exhyphenpenalty\z@
                \doublehyphendemerits\z@\finalhyphendemerits\z@
                \strut ##1\strut}%
        }%
        % If the linebreak is at a space, the latter will be displayed as visible
        % space at end of first line, and a continuation symbol starts next line.
        % Stretch/shrink are however usually zero for typewriter font.
        \def\FV@Space {%
            \nobreak\hskip\z@ plus\fontdimen3\font minus\fontdimen4\font
            \discretionary{\copy\Wrappedvisiblespacebox}{\Wrappedafterbreak}
            {\kern\fontdimen2\font}%
        }%
        
        % Allow breaks at special characters using \PYG... macros.
        \Wrappedbreaksatspecials
        % Breaks at punctuation characters . , ; ? ! and / need catcode=\active 	
        \OriginalVerbatim[#1,codes*=\Wrappedbreaksatpunct]%
    }
    \makeatother

    % Exact colors from NB
    \definecolor{incolor}{HTML}{303F9F}
    \definecolor{outcolor}{HTML}{D84315}
    \definecolor{cellborder}{HTML}{CFCFCF}
    \definecolor{cellbackground}{HTML}{F7F7F7}
    
    % prompt
    \makeatletter
    \newcommand{\boxspacing}{\kern\kvtcb@left@rule\kern\kvtcb@boxsep}
    \makeatother
    \newcommand{\prompt}[4]{
        {\ttfamily\llap{{\color{#2}[#3]:\hspace{3pt}#4}}\vspace{-\baselineskip}}
    }
    

    
    % Prevent overflowing lines due to hard-to-break entities
    \sloppy 
    % Setup hyperref package
    \hypersetup{
      breaklinks=true,  % so long urls are correctly broken across lines
      colorlinks=true,
      urlcolor=urlcolor,
      linkcolor=linkcolor,
      citecolor=citecolor,
      }
    % Slightly bigger margins than the latex defaults
    
    \geometry{verbose,tmargin=1in,bmargin=1in,lmargin=1in,rmargin=1in}
    
    

\begin{document}
    
    \maketitle
    
    

    
    \textbf{Problem 1}

\begin{enumerate}
\def\labelenumi{(\roman{enumi})}
\tightlist
\item
  \[f = 2x_{1}^{2} - 4x_1x_2 + 1.5x_2^2\] The stationary point of the
  function is obtained by taking the gradient and setting it to zero:\\
  \$ \rho =

  \begin{bmatrix}4x_1 - 4x_2 \\ -4x_1 + 3x_2 + 1 \end{bmatrix}

  \$ = 0\\
  Solving for x\_1 and x\_2 gives a stationary point (1,1).\\
  Indefiniteness of the Hessian can be shown by computing its
  eigenvalues:\\
\end{enumerate}

\$ H =

\begin{bmatrix} 4 & -4 \\ -4 & -1 \end{bmatrix}

\$\\
\$

\begin{vmatrix} 4 - \lambda & -4 \\ -4 & -1 - \lambda \end{vmatrix}

= 0 \$\\
\$ \lambda\emph{\{1\} = 10.54, \lambda}\{2\} = -4.54 \$\\
Since the eigenvalues are both positive and negative, the Hessian is
indefinite.\\
To find the direction of the slopes awy from the saddle point (1,1) we
need to find the points such that the difference f - f(1,1) \textless{}
0:\\
\$ f - f(1,1) = \frac{f_{x_1x_1}(1,1)(x_1 - 1)^2}{2} +
\frac{f_{x_1x_2}(1,1)(x_1 - 1)(x_2 - 1)}{2} +
\frac{f_{x_2x_2}(1,1)(x_2 - 1)^2}{2} \$\\
\$ f - 0.5 = 2(x\_1 - 1)\^{}2 - 2(x\_1 - 1)(x\_2 - 1) - 0.5(x\_2 -
1)\^{}2 \textless{} 0 \$

\textbf{Problem 2}\\
Part a: (Analytical) The problem asks to find a point closest to (-1, 0,
1) such that it is constrained to lie on the plane \$ x\_1 + 2x\_2 +
3x\_3 = 1 \$.\\
The problem can be formulated as: \$ \underset{x}{\operatorname{argmin}}
L(x\_0, x) \textbackslash{} \text{st } x\_1 + 2x\_2 + 3x\_3 = 1 \$\\
Using Lagrange multipliers on the equality constraint:\\
\$ \underset{x}{\operatorname{argmin}} L(x, x\_0) = (x\_\{01\} -
x\_1)\^{}2 + (x\_\{02\} - x\_2)\^{}2 + (x\_\{03\} - x\_3)\^{}2 =
\lambda g(x) \$\\
\$ \nabla L(x) = \lambda \nabla g(x) \$\\
\$ g(x) = c \$\\
The constraint equation works out to be:\\
\$ \nabla L(x) =

\begin{bmatrix}2x_1 + 2 \\ 2x_2 \\ 2x_3 - 2 \end{bmatrix}

=

\begin{bmatrix} 0 \\ \lambda \\ 2\lambda \end{bmatrix}

\$\\
Together with the equation \$ x\_1 + 2x\_2 + 3x\_3 - 1 = 0
\(, solving for the x's we get:\
\) x =

\begin{bmatrix} -1 \\ \frac{-1}{8} \\ \frac{3}{4} \end{bmatrix}

\$

\textbf{Problem 3}\\
(i) Let f(x) and g(x) be two convex functions defined on the convex set
X. The linear combination h(x) is defined as:\\
\$ h(x) = af(x) + bg(x) \$ If f(x) and g(x) are convex then:\\
\$ f(\lambda x\_1 + (1- \lambda)x\_2) \le \lambda f(x\_1) + (1 -
\lambda)f(x\_2) \$\\
\$ g(\lambda x\_1 + (1- \lambda)x\_2) \le \lambda g(x\_1) + (1 -
\lambda)g(x\_2) \$\\
\$ af(\lambda x\_1 + (1- \lambda)x\_2) + bg(\lambda x\_1 + (1-
\lambda)x\_2) \le a(\lambda f(x\_1) + (1 - \lambda)f(x\_2)) +
b(\lambda g(x\_1) + (1 - \lambda)g(x\_2)) \$\\
The right side can be rewritten as:\\
\$ \lambda(af(x\_1) + bg(x\_1)) + (1-\lambda)(af(x\_2) + bg(x\_2)) \$\\
From the definition of h(x), the right hand side can be transformed
to:\\
\$ \lambda h(x\_1) + (1-\lambda)h(x\_2) \$

\textbf{Problem 5}\\
(i) Formulation of optimization problem: \$
\underset{p_j}{\operatorname{min}} \sum\emph{\{k = 1\}\^{}\{n\} (I\_k -
I\_t)\^{}2 \$\\
st \$ 0 \le p\_j \le p}\{max\}\$\\
(ii) A formulation is convex if the objective is a convex function and
the constraints have a feasible set that is convex. Consider the
function: \$ f(p) = (a\_k\^{}Tp - I)\^{}2 \$\\
\$ \nabla f = 2a\_ka\_k\^{}Tp - 2Ia\_k \$\\
\$ \nabla \^{}2 f = 2a\_ka\_k\^{}T \$\\
Lemma: If \$ d\^{}THd \ge 0 \text{ for all d} \ne 0
\text{ then H is positive semi definite.} \$\\
In this case, applying the lemma gives us \$ d\^{}THd =
2d\textsuperscript{Ta\_ka\_k}Td
\(, which is always non negative. Moreover, the summation of k convex functions is also convex. Therefore the objective is convex. The feasible set from the constraints is convex because they form a segmented space that is a hypercube. \
(iii) If we require the power output of any of the n lamps to be less than p*, the modified optimization problem is: \
\) \underset{p_j}{\operatorname{min}} \sum\emph{\{k = 1\}\^{}\{n\} (I\_k
- I\_t)\^{}2 \$\\
st \$ 0 \le p\_j \le p}\{max\}\$ ,\\
\$ \sum\_\{j = 1\}\^{}\{k\} p\_j \textless{} p\^{}*, k =
\{1,\ldots,n\}\$\\
The second constraint can be further split into n separate constraints.
Unique solutions are possible if all the constraints are convex sets.
Consider the constraint \$ \sum\emph{\{j = 1\}\^{}\{l\} p\_j \lt p\^{}*
, l \epsilon k \$. The summation produces an l-dimensional half space of
a hyperplane, which is a convex set. Therefore the problem has a unique
solution.\\
(iv) If only half the lamps can be on at the same time, then the problem
has a combinatorial constraint, where out of n lamps, any n/2 of them
can be switched on. The constraint to be added to the formulation is: \$
\underset{p_j}{\operatorname{min}} \sum}\{k = 1\}\^{}\{n\} (I\_k -
I\_t)\^{}2 \$\\
st \$ 0 \le p\_j \le p\_\{max\}\$ ,\\
\$ \textbar\textbar x\textbar\textbar\_0 \le \frac{n}{2} \$\\
The l-0 norm is not convex, therefore the problem does not have a unique
solution.

    \begin{tcolorbox}[breakable, size=fbox, boxrule=1pt, pad at break*=1mm,colback=cellbackground, colframe=cellborder]
\prompt{In}{incolor}{1}{\boxspacing}
\begin{Verbatim}[commandchars=\\\{\}]
\PY{c+c1}{\PYZsh{} PROBLEM 2 : CODE FOR GRADIENT DESCENT}
\PY{n+nb}{print}\PY{p}{(}\PY{l+s+s2}{\PYZdq{}}\PY{l+s+s2}{Gradient descent solution}\PY{l+s+s2}{\PYZdq{}}\PY{p}{)}
\PY{k+kn}{import} \PY{n+nn}{numpy} \PY{k}{as} \PY{n+nn}{np}

\PY{k}{def} \PY{n+nf}{objective}\PY{p}{(}\PY{n}{x}\PY{p}{)}\PY{p}{:}
	\PY{k}{return} \PY{p}{(}\PY{l+m+mi}{2} \PY{o}{\PYZhy{}} \PY{p}{(}\PY{l+m+mi}{2}\PY{o}{*}\PY{n}{x}\PY{p}{[}\PY{l+m+mi}{0}\PY{p}{]}\PY{p}{)} \PY{o}{\PYZhy{}} \PY{p}{(}\PY{l+m+mi}{3}\PY{o}{*}\PY{n}{x}\PY{p}{[}\PY{l+m+mi}{1}\PY{p}{]}\PY{p}{)}\PY{p}{)}\PY{o}{*}\PY{o}{*}\PY{l+m+mi}{2} \PY{o}{+} \PY{p}{(}\PY{n}{x}\PY{p}{[}\PY{l+m+mi}{0}\PY{p}{]}\PY{p}{)}\PY{o}{*}\PY{o}{*}\PY{l+m+mi}{2} \PY{o}{+} \PY{p}{(}\PY{n}{x}\PY{p}{[}\PY{l+m+mi}{1}\PY{p}{]} \PY{o}{\PYZhy{}} \PY{l+m+mi}{1}\PY{p}{)}\PY{o}{*}\PY{o}{*}\PY{l+m+mi}{2}

\PY{k}{def} \PY{n+nf}{grad}\PY{p}{(}\PY{n}{x}\PY{p}{)}\PY{p}{:}
	\PY{k}{return} \PY{n}{np}\PY{o}{.}\PY{n}{asarray}\PY{p}{(}\PY{p}{[}\PY{l+m+mi}{10}\PY{o}{*}\PY{n}{x}\PY{p}{[}\PY{l+m+mi}{0}\PY{p}{]} \PY{o}{+} \PY{l+m+mi}{12}\PY{o}{*}\PY{n}{x}\PY{p}{[}\PY{l+m+mi}{1}\PY{p}{]} \PY{o}{\PYZhy{}} \PY{l+m+mi}{8}\PY{p}{,} \PY{l+m+mi}{12}\PY{o}{*}\PY{n}{x}\PY{p}{[}\PY{l+m+mi}{0}\PY{p}{]} \PY{o}{+} \PY{l+m+mi}{20}\PY{o}{*}\PY{n}{x}\PY{p}{[}\PY{l+m+mi}{1}\PY{p}{]} \PY{o}{\PYZhy{}} \PY{l+m+mi}{14}\PY{p}{]}\PY{p}{)}

\PY{k}{def} \PY{n+nf}{phi}\PY{p}{(}\PY{n}{a}\PY{p}{,}\PY{n}{x}\PY{p}{)}\PY{p}{:}
	\PY{n}{obj} \PY{o}{=} \PY{n}{objective}\PY{p}{(}\PY{n}{x}\PY{p}{)}
	\PY{n}{corr} \PY{o}{=} \PY{n}{a}\PY{o}{*}\PY{l+m+mf}{0.8}\PY{o}{*}\PY{n}{np}\PY{o}{.}\PY{n}{dot}\PY{p}{(}\PY{n}{grad}\PY{p}{(}\PY{n}{x}\PY{p}{)}\PY{p}{,} \PY{n}{grad}\PY{p}{(}\PY{n}{x}\PY{p}{)}\PY{p}{)}
	\PY{k}{return} \PY{n}{obj} \PY{o}{\PYZhy{}} \PY{n}{corr}

\PY{k}{def} \PY{n+nf}{line\PYZus{}search}\PY{p}{(}\PY{n}{x}\PY{p}{)}\PY{p}{:}
	\PY{n}{a} \PY{o}{=} \PY{l+m+mi}{1}  \PY{c+c1}{\PYZsh{} initialize step size}
	\PY{k}{while} \PY{n}{np}\PY{o}{.}\PY{n}{linalg}\PY{o}{.}\PY{n}{norm}\PY{p}{(}\PY{n}{phi}\PY{p}{(}\PY{n}{a}\PY{p}{,}\PY{n}{x}\PY{p}{)}\PY{p}{)}\PY{o}{\PYZlt{}}\PY{n}{objective}\PY{p}{(}\PY{n}{x}\PY{o}{\PYZhy{}}\PY{n}{a}\PY{o}{*}\PY{n}{grad}\PY{p}{(}\PY{n}{x}\PY{p}{)}\PY{p}{)}\PY{p}{:}  \PY{c+c1}{\PYZsh{} if f(x+a*d)\PYZgt{}phi(a) then backtrack. d is the search direction}
		\PY{n}{a} \PY{o}{=} \PY{l+m+mf}{0.5}\PY{o}{*}\PY{n}{a}
	\PY{k}{return} \PY{n}{a}

\PY{n}{eps} \PY{o}{=} \PY{l+m+mf}{1e\PYZhy{}3}
\PY{n}{x0} \PY{o}{=} \PY{p}{[}\PY{l+m+mi}{0}\PY{p}{,}\PY{l+m+mi}{0}\PY{p}{]}
\PY{n}{k} \PY{o}{=} \PY{l+m+mi}{0}
\PY{n}{x} \PY{o}{=} \PY{n}{x0}
\PY{n}{soln} \PY{o}{=} \PY{p}{[}\PY{n}{objective}\PY{p}{(}\PY{n}{x}\PY{p}{)}\PY{p}{]}

\PY{n}{error} \PY{o}{=} \PY{n}{np}\PY{o}{.}\PY{n}{linalg}\PY{o}{.}\PY{n}{norm}\PY{p}{(}\PY{n}{grad}\PY{p}{(}\PY{n}{x}\PY{p}{)}\PY{p}{)}

\PY{k}{while} \PY{n}{error} \PY{o}{\PYZgt{}}\PY{o}{=} \PY{n}{eps} \PY{o+ow}{and} \PY{n}{k} \PY{o}{\PYZlt{}} \PY{l+m+mi}{10000}\PY{p}{:}  \PY{c+c1}{\PYZsh{} keep searching while gradient norm is larger than eps}
	\PY{n}{a} \PY{o}{=} \PY{n}{line\PYZus{}search}\PY{p}{(}\PY{n}{x}\PY{p}{)}
	\PY{n}{x} \PY{o}{=} \PY{n}{x} \PY{o}{\PYZhy{}} \PY{n}{a}\PY{o}{*}\PY{n}{grad}\PY{p}{(}\PY{n}{x}\PY{p}{)}
	\PY{n}{error} \PY{o}{=} \PY{n}{np}\PY{o}{.}\PY{n}{linalg}\PY{o}{.}\PY{n}{norm}\PY{p}{(}\PY{n}{grad}\PY{p}{(}\PY{n}{x}\PY{p}{)}\PY{p}{)}
	\PY{n}{soln}\PY{o}{.}\PY{n}{append}\PY{p}{(}\PY{n}{objective}\PY{p}{(}\PY{n}{x}\PY{p}{)}\PY{p}{)}
	\PY{n}{k}\PY{o}{+}\PY{o}{=}\PY{l+m+mi}{1}
	\PY{k}{if} \PY{n}{k} \PY{o}{\PYZpc{}} \PY{l+m+mi}{1000} \PY{o}{==} \PY{l+m+mi}{0}\PY{p}{:} \PY{n+nb}{print}\PY{p}{(}\PY{l+s+s2}{\PYZdq{}}\PY{l+s+s2}{Iteration:}\PY{l+s+s2}{\PYZdq{}}\PY{p}{,} \PY{n}{k}\PY{p}{,} \PY{l+s+s2}{\PYZdq{}}\PY{l+s+s2}{alpha:}\PY{l+s+s2}{\PYZdq{}}\PY{p}{,} \PY{n}{a}\PY{p}{,} \PY{l+s+s2}{\PYZdq{}}\PY{l+s+s2}{Error:}\PY{l+s+s2}{\PYZdq{}}\PY{p}{,} \PY{n}{error}\PY{p}{,} \PY{l+s+s2}{\PYZdq{}}\PY{l+s+s2}{x:}\PY{l+s+s2}{\PYZdq{}}\PY{p}{,} \PY{n}{x}\PY{p}{)}

\PY{k+kn}{import} \PY{n+nn}{matplotlib}\PY{n+nn}{.}\PY{n+nn}{pyplot} \PY{k}{as} \PY{n+nn}{plt}
\PY{n}{y} \PY{o}{=} \PY{p}{[}\PY{n}{np}\PY{o}{.}\PY{n}{log}\PY{p}{(}\PY{n}{np}\PY{o}{.}\PY{n}{linalg}\PY{o}{.}\PY{n}{norm}\PY{p}{(}\PY{n}{s} \PY{o}{\PYZhy{}} \PY{n}{objective}\PY{p}{(}\PY{n}{x}\PY{p}{)} \PY{o}{+} \PY{l+m+mf}{1e\PYZhy{}6}\PY{p}{)}\PY{p}{)} \PY{k}{for} \PY{n}{s} \PY{o+ow}{in} \PY{n}{soln}\PY{p}{]}
\PY{n}{fig} \PY{o}{=} \PY{n}{plt}\PY{o}{.}\PY{n}{figure}\PY{p}{(}\PY{p}{)}
\PY{n}{plt}\PY{o}{.}\PY{n}{plot}\PY{p}{(}\PY{n}{y}\PY{p}{)}
\PY{n}{plt}\PY{o}{.}\PY{n}{xlabel}\PY{p}{(}\PY{l+s+s1}{\PYZsq{}}\PY{l+s+s1}{Iterations}\PY{l+s+s1}{\PYZsq{}}\PY{p}{)}
\PY{n}{plt}\PY{o}{.}\PY{n}{ylabel}\PY{p}{(}\PY{l+s+s1}{\PYZsq{}}\PY{l+s+s1}{log(|f\PYZhy{}f*|)}\PY{l+s+s1}{\PYZsq{}}\PY{p}{)}
\PY{n}{plt}\PY{o}{.}\PY{n}{title}\PY{p}{(}\PY{l+s+s1}{\PYZsq{}}\PY{l+s+s1}{Convergence plot \PYZhy{} Gradient Descent}\PY{l+s+s1}{\PYZsq{}}\PY{p}{)}
\PY{n}{x1} \PY{o}{=} \PY{l+m+mi}{1} \PY{o}{\PYZhy{}} \PY{l+m+mi}{2}\PY{o}{*}\PY{n}{x}\PY{p}{[}\PY{l+m+mi}{0}\PY{p}{]} \PY{o}{\PYZhy{}} \PY{l+m+mi}{3}\PY{o}{*}\PY{n}{x}\PY{p}{[}\PY{l+m+mi}{1}\PY{p}{]}
\PY{n}{x} \PY{o}{=} \PY{p}{[}\PY{n}{x1}\PY{p}{,} \PY{n}{x}\PY{p}{[}\PY{l+m+mi}{0}\PY{p}{]}\PY{p}{,} \PY{n}{x}\PY{p}{[}\PY{l+m+mi}{1}\PY{p}{]}\PY{p}{]}
\PY{n+nb}{print}\PY{p}{(}\PY{l+s+s2}{\PYZdq{}}\PY{l+s+s2}{Final solution:}\PY{l+s+s2}{\PYZdq{}}\PY{p}{,} \PY{n}{x}\PY{p}{)}

\PY{c+c1}{\PYZsh{}\PYZsh{}\PYZsh{}\PYZsh{}\PYZsh{}\PYZsh{}\PYZsh{}\PYZsh{}\PYZsh{}\PYZsh{}\PYZsh{}\PYZsh{}\PYZsh{}\PYZsh{}\PYZsh{}\PYZsh{}\PYZsh{}\PYZsh{}\PYZsh{}\PYZsh{}\PYZsh{}\PYZsh{}\PYZsh{}\PYZsh{}\PYZsh{}\PYZsh{}\PYZsh{}\PYZsh{}\PYZsh{}\PYZsh{}\PYZsh{}\PYZsh{}\PYZsh{}\PYZsh{}\PYZsh{}\PYZsh{}\PYZsh{}\PYZsh{}\PYZsh{}\PYZsh{}\PYZsh{}\PYZsh{}\PYZsh{}}
\PY{c+c1}{\PYZsh{} Problem 2: Newton\PYZsq{}s method}
\PY{n+nb}{print}\PY{p}{(}\PY{l+s+s2}{\PYZdq{}}\PY{l+s+s2}{Newton}\PY{l+s+s2}{\PYZsq{}}\PY{l+s+s2}{s method solution: }\PY{l+s+s2}{\PYZdq{}}\PY{p}{)}
\PY{k+kn}{import} \PY{n+nn}{numpy} \PY{k}{as} \PY{n+nn}{np}

\PY{k}{def} \PY{n+nf}{objective}\PY{p}{(}\PY{n}{x}\PY{p}{)}\PY{p}{:}
	\PY{k}{return} \PY{p}{(}\PY{l+m+mi}{2} \PY{o}{\PYZhy{}} \PY{p}{(}\PY{l+m+mi}{2}\PY{o}{*}\PY{n}{x}\PY{p}{[}\PY{l+m+mi}{0}\PY{p}{]}\PY{p}{)} \PY{o}{\PYZhy{}} \PY{p}{(}\PY{l+m+mi}{3}\PY{o}{*}\PY{n}{x}\PY{p}{[}\PY{l+m+mi}{1}\PY{p}{]}\PY{p}{)}\PY{p}{)}\PY{o}{*}\PY{o}{*}\PY{l+m+mi}{2} \PY{o}{+} \PY{p}{(}\PY{n}{x}\PY{p}{[}\PY{l+m+mi}{0}\PY{p}{]}\PY{p}{)}\PY{o}{*}\PY{o}{*}\PY{l+m+mi}{2} \PY{o}{+} \PY{p}{(}\PY{n}{x}\PY{p}{[}\PY{l+m+mi}{1}\PY{p}{]} \PY{o}{\PYZhy{}} \PY{l+m+mi}{1}\PY{p}{)}\PY{o}{*}\PY{o}{*}\PY{l+m+mi}{2}

\PY{k}{def} \PY{n+nf}{grad}\PY{p}{(}\PY{n}{x}\PY{p}{)}\PY{p}{:}
	\PY{k}{return} \PY{n}{np}\PY{o}{.}\PY{n}{asarray}\PY{p}{(}\PY{p}{[}\PY{l+m+mi}{10}\PY{o}{*}\PY{n}{x}\PY{p}{[}\PY{l+m+mi}{0}\PY{p}{]} \PY{o}{+} \PY{l+m+mi}{12}\PY{o}{*}\PY{n}{x}\PY{p}{[}\PY{l+m+mi}{1}\PY{p}{]} \PY{o}{\PYZhy{}} \PY{l+m+mi}{8}\PY{p}{,} \PY{l+m+mi}{12}\PY{o}{*}\PY{n}{x}\PY{p}{[}\PY{l+m+mi}{0}\PY{p}{]} \PY{o}{+} \PY{l+m+mi}{20}\PY{o}{*}\PY{n}{x}\PY{p}{[}\PY{l+m+mi}{1}\PY{p}{]} \PY{o}{\PYZhy{}} \PY{l+m+mi}{14}\PY{p}{]}\PY{p}{)}

\PY{k}{def} \PY{n+nf}{hessian}\PY{p}{(}\PY{p}{)}\PY{p}{:}
	\PY{k}{return} \PY{n}{np}\PY{o}{.}\PY{n}{asarray}\PY{p}{(}\PY{p}{[}\PY{p}{[}\PY{l+m+mi}{10}\PY{p}{,}\PY{l+m+mi}{12}\PY{p}{]}\PY{p}{,}\PY{p}{[}\PY{l+m+mi}{12}\PY{p}{,}\PY{l+m+mi}{20}\PY{p}{]}\PY{p}{]}\PY{p}{)}

\PY{k}{def} \PY{n+nf}{phi}\PY{p}{(}\PY{n}{a}\PY{p}{,}\PY{n}{x}\PY{p}{)}\PY{p}{:}
	\PY{n}{obj} \PY{o}{=} \PY{n}{objective}\PY{p}{(}\PY{n}{x}\PY{p}{)}
	\PY{n}{corr} \PY{o}{=} \PY{n}{a}\PY{o}{*}\PY{l+m+mf}{0.8}\PY{o}{*}\PY{n}{np}\PY{o}{.}\PY{n}{matmul}\PY{p}{(}\PY{n}{np}\PY{o}{.}\PY{n}{matmul}\PY{p}{(}\PY{n}{grad}\PY{p}{(}\PY{n}{x}\PY{p}{)}\PY{o}{.}\PY{n}{reshape}\PY{p}{(}\PY{l+m+mi}{1}\PY{p}{,}\PY{l+m+mi}{2}\PY{p}{)}\PY{p}{,}\PY{n}{np}\PY{o}{.}\PY{n}{linalg}\PY{o}{.}\PY{n}{inv}\PY{p}{(}\PY{n}{hessian}\PY{p}{(}\PY{p}{)}\PY{p}{)}\PY{p}{)}\PY{p}{,} \PY{n}{grad}\PY{p}{(}\PY{n}{x}\PY{p}{)}\PY{o}{.}\PY{n}{reshape}\PY{p}{(}\PY{l+m+mi}{2}\PY{p}{,}\PY{l+m+mi}{1}\PY{p}{)}\PY{p}{)}
	\PY{k}{return} \PY{n}{obj} \PY{o}{\PYZhy{}} \PY{n}{corr}

\PY{k}{def} \PY{n+nf}{line\PYZus{}search}\PY{p}{(}\PY{n}{x}\PY{p}{)}\PY{p}{:}
	\PY{n}{a} \PY{o}{=} \PY{l+m+mi}{1}  \PY{c+c1}{\PYZsh{} initialize step size}
	\PY{k}{while} \PY{n}{phi}\PY{p}{(}\PY{n}{a}\PY{p}{,}\PY{n}{x}\PY{p}{)}\PY{o}{\PYZlt{}}\PY{n}{objective}\PY{p}{(}\PY{n}{x}\PY{o}{\PYZhy{}}\PY{n}{a}\PY{o}{*}\PY{n}{grad}\PY{p}{(}\PY{n}{x}\PY{p}{)}\PY{p}{)}\PY{p}{:}  \PY{c+c1}{\PYZsh{} if f(x+a*d)\PYZgt{}phi(a) then backtrack. d is the search direction}
		\PY{n}{a} \PY{o}{=} \PY{l+m+mf}{0.5}\PY{o}{*}\PY{n}{a}
	\PY{k}{return} \PY{n}{a}

\PY{n}{eps} \PY{o}{=} \PY{l+m+mf}{1e\PYZhy{}3}
\PY{n}{x0} \PY{o}{=} \PY{p}{[}\PY{l+m+mi}{0}\PY{p}{,}\PY{l+m+mi}{0}\PY{p}{]}
\PY{n}{k} \PY{o}{=} \PY{l+m+mi}{0}
\PY{n}{x} \PY{o}{=} \PY{n}{np}\PY{o}{.}\PY{n}{asarray}\PY{p}{(}\PY{n}{x0}\PY{p}{)}\PY{o}{.}\PY{n}{reshape}\PY{p}{(}\PY{l+m+mi}{2}\PY{p}{,}\PY{l+m+mi}{1}\PY{p}{)}
\PY{n}{soln} \PY{o}{=} \PY{p}{[}\PY{n}{objective}\PY{p}{(}\PY{n}{x}\PY{p}{)}\PY{p}{]}

\PY{n}{lam} \PY{o}{=} \PY{n}{np}\PY{o}{.}\PY{n}{matmul}\PY{p}{(}\PY{n}{np}\PY{o}{.}\PY{n}{matmul}\PY{p}{(}\PY{n}{grad}\PY{p}{(}\PY{n}{x}\PY{p}{)}\PY{o}{.}\PY{n}{reshape}\PY{p}{(}\PY{l+m+mi}{1}\PY{p}{,}\PY{l+m+mi}{2}\PY{p}{)}\PY{p}{,}\PY{n}{np}\PY{o}{.}\PY{n}{linalg}\PY{o}{.}\PY{n}{inv}\PY{p}{(}\PY{n}{hessian}\PY{p}{(}\PY{p}{)}\PY{p}{)}\PY{p}{)}\PY{p}{,} \PY{n}{grad}\PY{p}{(}\PY{n}{x}\PY{p}{)}\PY{o}{.}\PY{n}{reshape}\PY{p}{(}\PY{l+m+mi}{2}\PY{p}{,}\PY{l+m+mi}{1}\PY{p}{)}\PY{p}{)}
\PY{k}{while} \PY{n}{lam}\PY{o}{*}\PY{o}{*}\PY{l+m+mi}{2}\PY{o}{/}\PY{l+m+mi}{2} \PY{o}{\PYZgt{}}\PY{o}{=} \PY{n}{eps} \PY{o+ow}{and} \PY{n}{k} \PY{o}{\PYZlt{}} \PY{l+m+mi}{10000}\PY{p}{:}  \PY{c+c1}{\PYZsh{} keep searching while gradient norm is larger than eps}
	\PY{n}{a} \PY{o}{=} \PY{n}{line\PYZus{}search}\PY{p}{(}\PY{n}{x}\PY{p}{)}
	\PY{n}{a} \PY{o}{=} \PY{l+m+mi}{1}
	\PY{n}{x} \PY{o}{=} \PY{n}{x} \PY{o}{\PYZhy{}} \PY{n}{a}\PY{o}{*}\PY{n}{np}\PY{o}{.}\PY{n}{matmul}\PY{p}{(}\PY{n}{np}\PY{o}{.}\PY{n}{linalg}\PY{o}{.}\PY{n}{inv}\PY{p}{(}\PY{n}{hessian}\PY{p}{(}\PY{p}{)}\PY{p}{)}\PY{p}{,} \PY{n}{grad}\PY{p}{(}\PY{n}{x}\PY{p}{)}\PY{o}{.}\PY{n}{reshape}\PY{p}{(}\PY{l+m+mi}{2}\PY{p}{,}\PY{l+m+mi}{1}\PY{p}{)}\PY{p}{)}
	\PY{n}{soln}\PY{o}{.}\PY{n}{append}\PY{p}{(}\PY{n}{objective}\PY{p}{(}\PY{n}{x}\PY{p}{)}\PY{p}{)}
	\PY{n}{lam} \PY{o}{=} \PY{n}{np}\PY{o}{.}\PY{n}{matmul}\PY{p}{(}\PY{n}{np}\PY{o}{.}\PY{n}{matmul}\PY{p}{(}\PY{n}{grad}\PY{p}{(}\PY{n}{x}\PY{p}{)}\PY{o}{.}\PY{n}{reshape}\PY{p}{(}\PY{l+m+mi}{1}\PY{p}{,}\PY{l+m+mi}{2}\PY{p}{)}\PY{p}{,}\PY{n}{np}\PY{o}{.}\PY{n}{linalg}\PY{o}{.}\PY{n}{inv}\PY{p}{(}\PY{n}{hessian}\PY{p}{(}\PY{p}{)}\PY{p}{)}\PY{p}{)}\PY{p}{,} \PY{n}{grad}\PY{p}{(}\PY{n}{x}\PY{p}{)}\PY{o}{.}\PY{n}{reshape}\PY{p}{(}\PY{l+m+mi}{2}\PY{p}{,}\PY{l+m+mi}{1}\PY{p}{)}\PY{p}{)}
	\PY{n}{k}\PY{o}{+}\PY{o}{=}\PY{l+m+mi}{1}
	\PY{k}{if} \PY{n}{k} \PY{o}{\PYZpc{}} \PY{l+m+mi}{1} \PY{o}{==} \PY{l+m+mi}{0}\PY{p}{:} \PY{n+nb}{print}\PY{p}{(}\PY{l+s+s2}{\PYZdq{}}\PY{l+s+s2}{Iteration:}\PY{l+s+s2}{\PYZdq{}}\PY{p}{,} \PY{n}{k}\PY{p}{,} \PY{l+s+s2}{\PYZdq{}}\PY{l+s+s2}{alpha:}\PY{l+s+s2}{\PYZdq{}}\PY{p}{,} \PY{n}{a}\PY{p}{,} \PY{l+s+s2}{\PYZdq{}}\PY{l+s+s2}{Error:}\PY{l+s+s2}{\PYZdq{}}\PY{p}{,} \PY{n}{error}\PY{p}{,} \PY{l+s+s2}{\PYZdq{}}\PY{l+s+s2}{x:}\PY{l+s+s2}{\PYZdq{}}\PY{p}{,} \PY{n}{x}\PY{p}{)}

\PY{k+kn}{import} \PY{n+nn}{matplotlib}\PY{n+nn}{.}\PY{n+nn}{pyplot} \PY{k}{as} \PY{n+nn}{plt}
\PY{n}{fig} \PY{o}{=} \PY{n}{plt}\PY{o}{.}\PY{n}{figure}\PY{p}{(}\PY{p}{)}

\PY{n}{y} \PY{o}{=} \PY{p}{[}\PY{n}{np}\PY{o}{.}\PY{n}{log}\PY{p}{(}\PY{n}{np}\PY{o}{.}\PY{n}{linalg}\PY{o}{.}\PY{n}{norm}\PY{p}{(}\PY{n}{s} \PY{o}{\PYZhy{}} \PY{n}{objective}\PY{p}{(}\PY{n}{x}\PY{p}{)} \PY{o}{+} \PY{l+m+mf}{1e\PYZhy{}6}\PY{p}{)}\PY{p}{)} \PY{k}{for} \PY{n}{s} \PY{o+ow}{in} \PY{n}{soln}\PY{p}{]}
\PY{n}{plt}\PY{o}{.}\PY{n}{plot}\PY{p}{(}\PY{n}{y}\PY{p}{)}
\PY{n}{plt}\PY{o}{.}\PY{n}{xlabel}\PY{p}{(}\PY{l+s+s1}{\PYZsq{}}\PY{l+s+s1}{Iterations}\PY{l+s+s1}{\PYZsq{}}\PY{p}{)}
\PY{n}{plt}\PY{o}{.}\PY{n}{ylabel}\PY{p}{(}\PY{l+s+s1}{\PYZsq{}}\PY{l+s+s1}{log(|f\PYZhy{}f*|)}\PY{l+s+s1}{\PYZsq{}}\PY{p}{)}
\PY{n}{plt}\PY{o}{.}\PY{n}{title}\PY{p}{(}\PY{l+s+s1}{\PYZsq{}}\PY{l+s+s1}{Convergence plot \PYZhy{} Newton}\PY{l+s+s1}{\PYZsq{}}\PY{p}{)}

\PY{n}{x1} \PY{o}{=} \PY{l+m+mi}{1} \PY{o}{\PYZhy{}} \PY{l+m+mi}{2}\PY{o}{*}\PY{n}{x}\PY{p}{[}\PY{l+m+mi}{0}\PY{p}{]} \PY{o}{\PYZhy{}} \PY{l+m+mi}{3}\PY{o}{*}\PY{n}{x}\PY{p}{[}\PY{l+m+mi}{1}\PY{p}{]}
\PY{n}{x} \PY{o}{=} \PY{p}{[}\PY{n}{x1}\PY{p}{,} \PY{n}{x}\PY{p}{[}\PY{l+m+mi}{0}\PY{p}{]}\PY{p}{,} \PY{n}{x}\PY{p}{[}\PY{l+m+mi}{1}\PY{p}{]}\PY{p}{]}
\PY{n+nb}{print}\PY{p}{(}\PY{l+s+s2}{\PYZdq{}}\PY{l+s+s2}{Final solution:}\PY{l+s+s2}{\PYZdq{}}\PY{p}{,} \PY{n}{x}\PY{p}{)}
\end{Verbatim}
\end{tcolorbox}

    \begin{Verbatim}[commandchars=\\\{\}]
Gradient descent solution
Final solution: [-1.0713988626792919, -0.14249520055213166, 0.7854630879278517]
Newton's method solution:
Iteration: 1 alpha: 1 Error: 0.000910695792946387 x: [[-0.14285714]
 [ 0.78571429]]
Final solution: [array([-1.07142857]), array([-0.14285714]),
array([0.78571429])]
    \end{Verbatim}

    \begin{center}
    \adjustimage{max size={0.9\linewidth}{0.9\paperheight}}{Homework2_files/Homework2_1_1.png}
    \end{center}
    { \hspace*{\fill} \\}
    
    \begin{center}
    \adjustimage{max size={0.9\linewidth}{0.9\paperheight}}{Homework2_files/Homework2_1_2.png}
    \end{center}
    { \hspace*{\fill} \\}
    

    % Add a bibliography block to the postdoc
    
    
    
\end{document}
